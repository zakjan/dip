\begin{introduction}

At its core, the Internet is built on top of IP and TCP protocols, which are used to package data into small packets for transport. As these packets travel across the world, they cross many computer systems in many countries. Because the core protocols don't provide any security by themselves, anyone with access to the communication links can gain full access to the data as well as change the traffic without detection.

Over the last years, the Internet has grown into a major platform for the world's communication. The Internet's trustworthiness has become critical to its success. If a person cannot trust that they are communicating with the party they intend, they won't give out their confidential data. If they cannot be assured that delivered information isn't modified in transit, they won't trust it as much.

The important properties of confidentiality, authentication and integrity are currently best provided on the Internet by the TLS protocol. The HTTP protocol implements it in its secure HTTPS variant, which means using \textit{"https://"} URLs. In the past, websites have deployed HTTPS rarely, often only when financial transactions take place.

More recently, however, it has become apparent that nearly all activity on the Internet can be considered sensitive. If a third party can modify content in transit, the power of the Internet can easily be turned against the user. For example, internet providers can "harmlessly" insert advertisements into websites. More hostile attacks include editing crucial information on the website, or injecting malware.

The TLS protocol does not dictate which cryptographic algorithms need to be used. Instead, TLS serves as a framework establishing and maintaining a secure comminucation channel suitable for sending sensitive messages, while new cryptographic algorithms can be implemented using a common interface.

Currently the TLS protocol uses a MAC-then-Encrypt generic composition to achieve both confidentiality and integrity goals. More recently, the idea of providing both confidentiality and integrity goals using a single cryptosystem has become accepted. In this concept, the combination of encryption and authentication algorithm is replaced by a single authenticated encryption algorithm, such as AES-GCM.

In 2013, CAESAR was announced. It is a worldwide cryptographic competition, focused on finding new methods of authenticated encryption, that offer advantages against commonly used AES-GCM and will be suitable for widespread adoption. Submitted algorithms will be publicly evaluated by committee of researchers in fields of cryptography and cryptoanalysis.



\end{introduction}
