\chapter{Competition for Authenticated Encryption: Security, Applicability and Robustness}

In 2013, CAESAR was announced. It is a worldwide cryptographic competition, focused on finding new methods of authenticated encryption, that offer advantages against commonly used AES-GCM and will be suitable for widespread adoption. Submitted algorithms will be publicly evaluated by committee of researchers in fields of cryptography and cryptoanalysis.

This competition follows a long tradition of focused competitions in secret-key cryptography:

\begin{itemize}
  \item In 1997, United States National Institute of Standards and Technology (NIST) announced an open competition for a new symmetric cupher, Advanced Encryption Standard (AES). This competition attracted 15 submissions from 50 cryptographers around the world. In the end, Rijndael was chosen as AES.
  \item In 2004, European Network of Excellence in Cryptology (ECRYPT) announced the ECRYPT Stream Cipher Project (eSTREAM), a call for new stream ciphers suitable for widespread adoption. This call attracted 34 submissions from 100 cryptographers around the world. In the end, the eSTREAM committee selected a portfolio containing several stream ciphers.
  \item In 2007, NIST announced an open competition for a new hash standard to Secure Hash Algorithm family (SHA-3). This competition attracted 64 submissions from 200 cryptographers around the world. In the end, Keccak was chosen as SHA-3.
\end{itemize}

All past cryptographic competitions attracted many submissions from cryptographers around the world, and then even more security and performance evaluations from cryptanalysts. They are generally viewed as having provided a tremendous boost to the cryptographic research community's understanding of underlying concepts, and a tremendous increase in confidence in the security of some existing cryptosystems. Similar comments are expected to apply to CAESAR. \cite{crypto-competitions}



\section{Functional requirements}

For the purpose of CAESAR competition, an \textit{authenticated cipher} is a pair of encrypt and decrypt functions, meeting the following specific requirements.

All inputs and outputs should be represented as opaque byte-strings (members of a set $\mathbb{Z}_{2^8}^*$), because they benefit from direct support of current computers to store and transmit them.

A cipher is permitted to be defined using objects other then byte-strings, nevertheless it must specify an unambiguous relationship between those objects and byte-strings (e.g. endianness of integers).

A cipher must specify a length of all fixed-length inputs. It is permitted to specify a maximum length of various-length inputs, but this limit must not be smaller than 65 kB and submissions are expected to include justification for any maximum length limits.

No other restrictions on their structure should be imposed, all inputs meeting the length restrictions must be accepted.

\subsection{Inputs and outputs}

\begin{table}
  \centering

  \begin{tabular}{lllll}
    \csvtable{tables/caesar-inputs.csv}
      { & required & integrity & confidentiality & single-use}
      {\csvcoli & \yesbf{\csvcolii} & \yesbf{\csvcoliii} & \yesbf{\csvcoliv} & \yesbf{\csvcolv}}
  \end{tabular}

  \caption{CAESAR inputs}
  \label{table/caesar-inputs}
\end{table}


A \textit{plaintext} is a variable-length input/output, a piece of confidential information a sender wants to transmit to a receiver, as introduced in \autoref{toc/encryption}.

A \textit{ciphertext} is a variable-length input/output counterpart of the plaintext, that can be transmitted over an insecure channel. it is usually longer then the plaintext, because it contains an \textit{authentication tag}. This length difference is permitted to be fixed constant, thus leaking the plaintext length via the ciphertext length. Designers are advised that minimizing ciphertext length is generally considered more valuable than hiding plaintext length.

A \textit{key} is a fixed-length input, which determines the output of both encrypt and decrypt functions. The key must be shared between both communicating parties prior to encrypted communication. Without a key or with a different one then used in the encrypt function, the decrypt function produces no useful result. This follows the Kerckhoffs' principle as introduced in \autoref{toc/kerckhoffs-principle}.

An \textit{associated data} is a variable-length input, a piece of information known by both communicating parties, which doesn't need to meet confidentiality requirement. However, its origin still needs to be verified by the receiving party. It can be for example some message metadata, such as version of used protocol.

A \textit{nonce} (number used once) is a fixed-length input. It is a public value, which which is usually used as IV for the enclosed cipher. Such IVs should be unique for each encryption run, so it makes all ciphertexts undistinguishable even if the same key, message and associated data is used.

However, CAESAR call for submissions requests an unusual authenticated encryption interface. The user, who wants to encrypt, instead of providing the usual \textit{four} arguments (the key, nonce, associated data, and message) for authenticated encryption, he needs to provide \textit{five} arguments. The nonce has been transformed into a \textit{public message number} and \textit{secret message number}. \cite{cryptoeprint:2013:242}

A \textit{public message number} is a fixed-length input. It is a public value with the same requirements as the nonce in the original definition of authenticated encryption.

A \textit{secret message number} is a fixed-length input. It is a secret value, recoverable from the ciphertext, however it is not a part of the plaintext. Allowing both a secret message number and a public message number creates possibilities of different levels of their security requirements.

All inputs must meet various security purposes, as indicated by \autoref{table/caesar-inputs}.


\section{Software requirements}
\label{toc/caesar-api}

Each first-round submission must contain a portable reference software implementation to support public understanding of the cipher, cryptanalysis, verification of subsequent implementations, etc. The implementation must cover all recommended parameter sets, and must compute exactly the function specified in the submission. The reference implementation is expected to be optimized for clarity, not for performance. \cite{crypto-competitions}

The submission must export the following constants:

\begin{itemize}
  \item \texttt{CRYPTO\_KEYBYTES} -- the fixed length of key
  \item \texttt{CRYPTO\_NSECBYTES} -- the fixed length of secret message number (0 if not supported)
  \item \texttt{CRYPTO\_NPUBBYTES} -- the fixed length of public message number (0 if not supported)
  \item \texttt{CRYPTO\_ABYTES} -- the maximum (usually fixed) length difference between plaintext and ciphertext
\end{itemize}

\inputminted{c}{code/caesar/constants.c}

The submission must export the following \texttt{crypto\_aead\_encrypt} and \texttt{crypto\_aead\_decrypt} functions, which perform the encrypt and decrypt operation respectively.

\inputminted{c}{code/caesar/encrypt.c}
\inputminted{c}{code/caesar/decrypt.c}

The output of functions must be determined entirely by the inputs in their arguments and must not be affected by any randomness or other hidden inputs.

The functions should perform the operation in constant time with regard to any input data (even invalid data) to prevent timing side-channel attacks.

The decrypt function must return -1 if the ciphertext is not valid, i.e. if the ciphertext does not equal the encryption of any (plaintext, secret message number) pair with this associated data, public message number, and secret key. The functions may return other negative numbers to indicate other failures, for example memory-allocation failures. \cite{crypto-competitions}


\section{Hardware requirements}

Each submission selected for the second round will also be required to include a reference hardware design (i.e., a reference implementation in the VHDL or Verilog languages). Details of the hardware API have not yet been specified. \cite{crypto-competitions}


\section{Submissions}

The competition was announced on 2013-01-15 at the Early Symmetric Crypto workshop in Mondorf-les-Bains, also announced online. First-round submission PDFs must have been received till 2014-03-15, reference software implementations must have been received till 2014-04-15.

After passing the first-round deadline, all submissions were published. Announcement of second-round candidates was initially scheduled for 2015-01-15. However it is hard work to do a proper security review of all submissions. Currently in the time of writing my thesis, the second-round candidates announcement is being postponed every month.

All submission PDFs can be downloaded on CAESAR homepage\footnote{\url{http://competitions.cr.yp.to/caesar-submissions.html}}. Submission source codes are bundled together with Supercop benchmark application\footnote{\url{http://bench.cr.yp.to/supercop.html}}.

Also there was \textit{Directions in Authenticated Ciphers} (DIAC) 2014 conference, where a lot of sumbission authors presented their candidates. Slides of their talks are available for download on the DIAC website\footnote{\url{http://2014.diac.cr.yp.to/index.html}}.

There were submitted 57 candidates, till now 9 of them were withdrawn by their authors, so there are 48 candidates remaining. It is expected that the announcement of second-round candidates will contain about a half of them.

\todo{analyzovat každou podrobně je nad rámec této práce}

\todo{the OCB mode, the GCM mode, the duplex sponge or AES-based block-cipher/permutation, Keccak-based permutation, stream-cipher based permutation}

\subsection{Types}

\subsubsection{Duplex sponge functions}

Sponge as a design tool

On top of its original goal as a security reference, we realized that the sponge construction could also be used to build efficient cryptographic primitives. An important aspect is that the cryptographic primitive to be designed is a fixed-length permutation rather than harder-to-build structures such as block ciphers or dedicated compression functions. This is rather good news in itself, as all the symmetric cryptographic services can be realized using only a single primitive: a fixed-length permutation. As opposed to a block cipher, a fixed-length permutation makes no distinction between data and key input and hence can treat all input bits on an equal footing and at the same time can be made simpler.

\url{http://sponge.noekeon.org/}

The first 128-bit message block is handled directly, and taking in account the tag generation one needs m + 1 internal cipher calls to process messages of m block of n bits each. This is particularly important in many lightweight applications where message sent are usually composed of a few dozens of bytes (this is common disadvantage of sponge-based or stream cipher based lightweight designs like FIDES [2]) or ALE [8]).

Deoxys

Guido Bertoni, Joan Daemen, Michael Peeters, and Gilles Van Assche. Duplexing the Sponge: Single-Pass
Authenticated Encryption and Other Applications. In Ali Miri and Serge Vaudenay, editors, Selected Areas
in Cryptography, volume 7118 of LNCS, pages 320–337. Springer, 2011.

\subsubsection{Block modes}

It turns out that it is quite difficult to construct a secure lightweight authenticated cipher. Hence, it is meaningful to develop a secure lightweight authenticatedvencryption mode so that the previous designs of lightweight block ciphers can be converted to lightweight authenticated ciphers.

\subsection{Primitives}

AES

ARX - addition, rotation, XOR

LRX - logic, rotation, XOR

quaternions

\subsection{Selection criteria}

\subsubsection{Small messages}

It performs very well for small messages (only m + 1 calls to Joltik-BC are required for a m
block message and without any precomputation), in contrary to sponge or stream cipher based
lightweight designs that require a strong initialization stage. Such a feature is very important
as many constrained environments will only cipher very short messages (for example a 96-bit
Electronic Product Code)

Joltik

\begin{description}
  \item[Online (one-pass)]
\end{description}


\todo{popis algoritmů byly v soutěži, v čem se lišily, jaké a jak v nich byly nalezeny zranitelnosti a proto nepostoupily}

\todo{výběr algoritmu pro implementaci}




\section{NORX}

NO(T A)RX

ARX - Addition, Rotation, XOR

\subsection{Design goals}

\begin{itemize}
  \item \textbf{High security}
  \item \textbf{High speed} (in SW \textit{and} HW)
  \item \textbf{Simplicity} (of spec \textit{and} code)
  \item Online / one-pass
  \item Scalability (parallelism, unrolling)
  \item High key agility (no "key schedule")
  \item Side-channel leaks robustness (esp. timings)
\end{itemize}

\subsection{Parameters}

\begin{description}
  \item[Word Bit Size] $W \in {32, 64}$
  \item[Number of Rounds] $1 \leq R \leq 63$
  \item[Parallelism Degree] $0 \leq D \leq 255$ (0?)
  \item[Tag Bit Size] $|A| \leq 10W$
\end{description}

\begin{table}
  \centering

  \begin{tabular}{llll}
    \csvtable{tables/norx-proposed-instances.csv}
      {NORX$W$-$R$-$D$ & Nonce ($2W$) & Key ($4W$) & Tag ($4W$)}
      {\texttt{\csvcoli} & \csvcolii & \csvcoliii & \csvcoliv}
  \end{tabular}

  \caption{NORX proposed instances}
\end{table}


R=6: higher security margin

D=4: high throughput on parallel architectures

\section{Selection}
\label{toc/caesar-selection}

Because in the time of writing this thesis the announcement of second-round candidates is still being postponed, I could not choose a qualified candidate, which I would implement into OpenSSL. So I decided to implement a cipher using the generic CAESAR API (see \autoref{toc/caesar-api}).

As a reference cipher for my implementation part I chose the NORX cipher, because it have received no negative analysis. However it is not important which particular cipher I used, because the cipher can be easily switched for a different cipher complying with the CAESAR API, as soon as the the second-round candidate or final announcement will be made.

