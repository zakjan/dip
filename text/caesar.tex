\chapter{Competition for Authenticated Encryption: Security, Applicability and Robustness}

CAESAR is worldwide cryptographic competition, focused on finding new methods of authenticated encryption, that offer advantages against commonly used AES-GCM and will be suitable for widespread adoption. Submitted algorithms will be publicly evaluated by committee of researchers in fields of cryptography and cryptoanalysis.

\todo{popis algoritmů byly v soutěži, v čem se lišily, jaké a jak v nich byly nalezeny zranitelnosti a proto nepostoupily}

\todo{výběr algoritmu pro implementaci}

\section{Submission requirements}

An \textbf{authenticated cipher} is a function with five byte-string inputs and one byte-string output. The five inputs are:

\begin{description}
  \item[plaintext] variable-length
  \item[associated data] variable-length
  \item[secret message number] fixed-length
  \item[public message number] fixed-length
  \item[key] fixed-length
\end{description}

The first four inputs have different security purposes, as indicated by \autoref{table/caesar-inputs}.

\begin{table}
  \centering

  \begin{tabular}{lllll}
    \csvtable{tables/caesar-inputs.csv}
      { & required & integrity & confidentiality & single-use}
      {\csvcoli & \yesbf{\csvcolii} & \yesbf{\csvcoliii} & \yesbf{\csvcoliv} & \yesbf{\csvcolv}}
  \end{tabular}

  \caption{CAESAR inputs}
  \label{table/caesar-inputs}
\end{table}



The output is a variable-length ciphertext. The first four inputs have different security purposes, as indicated by the following table:

\section{C API}

\begin{lstlisting}
#define CRYPTO_KEYBYTES 16
#define CRYPTO_NSECBYTES 0
#define CRYPTO_NPUBBYTES 12
#define CRYPTO_ABYTES 16

#include "crypto_aead.h"

int crypto_aead_encrypt(
unsigned char *c,unsigned long long *clen,
const unsigned char *m,unsigned long long mlen,
const unsigned char *ad,unsigned long long adlen,
const unsigned char *nsec,
const unsigned char *npub,
const unsigned char *k
)
{
...
... the code for the cipher implementation goes here,
... generating a ciphertext c[0],c[1],...,c[*clen-1]
... from a plaintext m[0],m[1],...,m[mlen-1]
... and associated data ad[0],ad[1],...,ad[adlen-1]
... and secret message number nsec[0],nsec[1],...
... and public message number npub[0],npub[1],...
... and secret key k[0],k[1],...
...
return 0;
}

int crypto_aead_decrypt(
unsigned char *m,unsigned long long *mlen,
unsigned char *nsec,
const unsigned char *c,unsigned long long clen,
const unsigned char *ad,unsigned long long adlen,
const unsigned char *npub,
const unsigned char *k
)
{
...
... the code for the cipher implementation goes here,
... generating a plaintext m[0],m[1],...,m[*mlen-1]
... and secret message number nsec[0],nsec[1],...
... from a ciphertext c[0],c[1],...,c[clen-1]
... and associated data ad[0],ad[1],...,ad[adlen-1]
... and public message number npub[0],npub[1],...
... and secret key k[0],k[1],...
...
return 0;
}
\end{lstlisting}

npub - IV
nsec - nepoužito

\section{Submissions}

analyzovat každou podrobně je nad rámec této práce

the OCB mode, the GCM mode, the duplex sponge or AES-based block-cipher/permutation, Keccak-based permutation, stream-cipher based permutation

\section{Types}

\subsection{Duplex sponge functions}

Sponge as a design tool

On top of its original goal as a security reference, we realized that the sponge construction could also be used to build efficient cryptographic primitives. An important aspect is that the cryptographic primitive to be designed is a fixed-length permutation rather than harder-to-build structures such as block ciphers or dedicated compression functions. This is rather good news in itself, as all the symmetric cryptographic services can be realized using only a single primitive: a fixed-length permutation. As opposed to a block cipher, a fixed-length permutation makes no distinction between data and key input and hence can treat all input bits on an equal footing and at the same time can be made simpler.

\url{http://sponge.noekeon.org/}

The first 128-bit message block is handled directly, and taking in account the tag generation one needs m + 1 internal cipher calls to process messages of m block of n bits each. This is particularly important in many lightweight applications where message sent are usually composed of a few dozens of bytes (this is common disadvantage of sponge-based or stream cipher based lightweight designs like FIDES [2]) or ALE [8]).

Deoxys

Guido Bertoni, Joan Daemen, Michael Peeters, and Gilles Van Assche. Duplexing the Sponge: Single-Pass
Authenticated Encryption and Other Applications. In Ali Miri and Serge Vaudenay, editors, Selected Areas
in Cryptography, volume 7118 of LNCS, pages 320–337. Springer, 2011.

\subsection{Block modes}

It turns out that it is quite difficult to construct a secure lightweight authenticated cipher. Hence, it is meaningful to develop a secure lightweight authenticatedvencryption mode so that the previous designs of lightweight block ciphers can be converted to lightweight authenticated ciphers.

\section{Primitives}

AES

ARX - addition, rotation, XOR

LRX - logic, rotation, XOR

quaternions

\section{Selection criteria}

\subsection{Small messages}

It performs very well for small messages (only m + 1 calls to Joltik-BC are required for a m
block message and without any precomputation), in contrary to sponge or stream cipher based
lightweight designs that require a strong initialization stage. Such a feature is very important
as many constrained environments will only cipher very short messages (for example a 96-bit
Electronic Product Code)

Joltik

\begin{description}
  \item[Online (one-pass)]
\end{description}




\section{NORX}

NO(T A)RX

ARX - Addition, Rotation, XOR

\subsection{Design goals}

\begin{itemize}
  \item \textbf{High security}
  \item \textbf{High speed} (in SW \textit{and} HW)
  \item \textbf{Simplicity} (of spec \textit{and} code)
  \item Online / one-pass
  \item Scalability (parallelism, unrolling)
  \item High key agility (no "key schedule")
  \item Side-channel leaks robustness (esp. timings)
\end{itemize}

\subsection{Parameters}

\begin{description}
  \item[Word Bit Size] $W \in {32, 64}$
  \item[Number of Rounds] $1 \leq R \leq 63$
  \item[Parallelism Degree] $0 \leq D \leq 255$ (0?)
  \item[Tag Bit Size] $|A| \leq 10W$
\end{description}

\begin{table}
  \centering

  \begin{tabular}{llll}
    \csvtable{tables/norx-proposed-instances.csv}
      {NORX$W$-$R$-$D$ & Nonce ($2W$) & Key ($4W$) & Tag ($4W$)}
      {\texttt{\csvcoli} & \csvcolii & \csvcoliii & \csvcoliv}
  \end{tabular}

  \caption{NORX proposed instances}
\end{table}


R=6: higher security margin

D=4: high throughput on parallel architectures

\section{Selection}
\label{toc/caesar-selection}

Because in the time of writing this thesis the announcement of second-round candidates is still being postponed, I could not choose a qualified candidate, which I would implement into OpenSSL. So I decided to implement a cipher using the generic CAESAR API (see \autoref{toc/caesar-api}).

As a reference cipher for my implementation part I chose the NORX cipher, because it have received no negative analysis. However it is not important which particular cipher I used, because the cipher can be easily switched for a different cipher complying with the CAESAR API, as soon as the the second-round candidate or final announcement will be made.

