\section{libcrypto library}
\label{toc/openssl-libcrypto}

The libcrypto\footnote{\url{https://wiki.openssl.org/index.php/Libcrypto_API}} library provides high-level and low-level interfaces for the implemented fundamental cryptographic algorithms.

For most uses, users should use the high-level interface that is provided for performing cryptographic operations. This is known as the EVP\footnote{\url{https://wiki.openssl.org/index.php/EVP}} interface (short for Envelope). This interface provides a suite of functions for performing encryption/decryption (both symmetric and asymmetric), signing/verifying, as well as generating hashes and MAC codes, across the full range of OpenSSL supported algorithms and modes. Working with the high-level interface means that a lot of the complexity of performing cryptographic operations is hidden from view. A single consistent API is provided. In the event that you need to change your code to use a different algorithm (for example), then this is a simple change when using the high-level interface. In addition low-level issues such as padding and encryption modes are all handled for you.

The EVP functions provide a high-level interface to OpenSSL cryptographic functions. They provide the following features:

\begin{itemize}
  \item A single consistent interface regardless of the underlying algorithm or mode
  \item Support for an extensive range of algorithms
  \item Encryption/Decryption using both symmetric and asymmetric algorithms
  \item Sign/Verify
  \item Key derivation
  \item Secure Hash functions
  \item Message Authentication Codes
  \item Support for external crypto engines
\end{itemize}

In addition to the high-level interface, OpenSSL also provides low-level interfaces for working directly with the individual algorithms. These low-level interfaces are not recommended for the novice user, but provide a degree of control that may not be possible when using only the high-level interface.

Where possible, the high-level EVP interface should be used in preference to the low-level interfaces. This is because the code then becomes transparent to the algorithm used and much more flexible. Additionally, the EVP interface will ensure the use of platform specific cryptographic acceleration such as AES-NI. The low-level interfaces do not provide the guarantee.


\subsection{EVP API}
\label{toc/openssl-evp}

A specific cipher or message digest algorithm is identified by an unique \texttt{EVP\_CIPHER} or \texttt{EVP\_MD} struct respectively. OpenSSL user is not expected to create these themself, but instead they can use a built-in function to return the struct of particular algorithm that they wish to use. In the following text I focus only on cipher algorithms.

An extract from \texttt{evp.h} header file listing some functions returning \texttt{EVP\_CIPHER} struct is shown below. There is a specific function for every supported combination of cipher algorithm and its parameters (a key length, a block mode, etc.).

\inputminted{c}{code/openssl-evp-ciphers.c}

The \texttt{EVP\_CIPHER} is represented by the following signature:

\inputminted{c}{code/openssl-evp-cipher.c}

The meaning of its important parameters follows:

\begin{itemize}
  \item \texttt{nid} -- integer; unique identifier
  \item \texttt{block\_size} -- integer; cipher block size
  \item \texttt{key\_len} -- integer; cipher key length
  \item \texttt{iv\_len} -- integer; cipher IV length
  \item \texttt{flags} -- bit array, represented as integer; cipher flags specifying its capabilities
  \item \texttt{init} -- function pointer; called during initialization, it can allocate the cipher context before the operation
  \item \texttt{do\_cipher} -- function pointer; called during sending the input data, it should perform the encrypt/decrypt operation
  \item \texttt{cleanup} -- function pointer; called during finalization, it can free the cipher context after the operation
  \item \texttt{ctx\_size} -- integer; cipher-specific context size, can be used to store cipher-specific data during the operation
  \item \texttt{ctrl} -- function pointer; used to invoke special actions, which do not have a specific field in the \texttt{EVP\_CAESAR} struct
\end{itemize}

Prior to performing the encrypt/decrypt operation, a cipher context must be allocated and initialized to store setting and state during the operation. The cipher context is represented by \texttt{EVP\_CIPHER\_CTX} struct.

\inputminted{c}{code/openssl-evp-cipher-ctx-new.c}

Note the cipher context is different from cipher-specific context. The cipher-specific context is stored inside of the cipher context, however the cipher-specific context is allocated and freed by the cipher itself in its \texttt{init} and \texttt{cleanup} functions, which are called by EVP internally, and a user does not need to care about it.

So now we are prepared to perform the encrypt/decrypt operation. First, we need to set a cipher, a secret key and an initialization vector (IV). We can use an initialization function, specifically named \texttt{EVP\_EncryptInit\_ex} or \texttt{EVP\_DecryptInit\_ex}. There is also an universal \texttt{EVP\_CipherInit\_ex} function with \texttt{enc} parameter, which controls which operation is performed. However I recommend the first option, because usually we are sure which operation do we want to perform, so it should be hardcoded using the specific functions instead of depending of an integer parameter. It internally calls the cipher's \texttt{init} function from the \texttt{EVP\_CIPHER} struct.

\inputminted{c}{code/openssl-evp-cipherinit.c}

Any input data should be sent to the cipher by an update function \texttt{EVP\_EncryptUpdate} or \texttt{EVP\_DecryptUpdate}. It accepts the input data, and it fills the provided output buffer with the encrypted/decrypted output data, and returns a number of written bytes. Usualy it can be called multiple times, if the cipher supports streaming.

The functions internally calls the cipher's \texttt{do\_cipher} function with the same parameters.

\inputminted{c}{code/openssl-evp-cipherupdate.c}

After the input data block ends, a finalization function \texttt{EVP\_EncryptFinal\_ex} or \texttt{EVP\_DecryptFinal\_ex} needs to be called. This function can append few more bytes to output, for example a padding or an authentication tag.

The functions internally call the cipher's \texttt{do\_cipher} function with null parameters to signalize the end of operation. The cipher can react by appending few more bytes to output, for example a padding or an authentication tag.

\inputminted{c}{code/openssl-evp-cipherfinal.c}

After all cipher operations were finished, the cipher context must be cleaned up and freed by the \texttt{EVP\_CIPHER\_CTX\_free} function. It internally calls the cipher's \texttt{cleanup} function.

\inputminted{c}{code/openssl-evp-cipher-ctx-free.c}


\subsection{EVP API -- Symmetric encryption and decryption}
\label{toc/openssl-evp-encryption}

With the knowledge of EVP API as described in \autoref{toc/openssl-evp}, a user can perform symmetric encryption and decryption operations across a wide range of algorithms and modes. The following code shows how to use it to encrypt and decrypt a piece of confidential information.

Encryption using the EVP API consists of the following stages:

\begin{itemize}
  \item Setting up a cipher context
  \item Initializing the encryption operation, providing key, IV
  \item Providing plaintext bytes to be encrypted
  \item Finalizing the encryption operation
\end{itemize}

The sample \texttt{encrypt} function uses AES-128 in CBC mode. It takes as arguments the plaintext, the length of the plaintext, the key, and the IV. Also it takes in a buffer to put the ciphertext in (which we assume to be long enough), and will return the length of the ciphertext that it writtes.

The length of plaintext is necessary, OpenSSL cannot use \texttt{strlen} function to determine its length, because it can contain any data, even null (\texttt{\textbackslash0}) bytes. The length of key and IV is fixed, appropriate for the chosen cipher, which means both the key and IV 16 bytes long for AES-128 in CBC mode.

The source code follows. It misses the most of error handling code, which would be necessary in a real application.

\inputminted{c}{code/openssl-evp-encrypt.c}

Decryption using the EVP API consists of the following steps:

\begin{itemize}
  \item Setting up a cipher context
  \item Initializing the decryption operation, providing key, IV
  \item Providing ciphertext bytes to be decrypted
  \item Finalizing the decryption operation
\end{itemize}

The sample \texttt{decrypt} function uses AES-128 in CBC mode. It takes almost the same arguments as the \texttt{encrypt} function, with the exception that the plaintext and the ciphertext are swapped.

The source code follows. It misses the most of error handling code, which would be necessary in a real application.

\inputminted{c}{code/openssl-evp-decrypt.c}


\subsection{EVP API -- Authenticated encryption and decryption}
\label{toc/openssl-evp-aead-encryption}

Following the recent advances in AEAD, the EVP API of libcrypto library also supports the ability to perform authenticated encryption and decryption. It provides confidentiality by encrypting the data, and authenticity by creating a MAC tag over the encrypted data.

Using AEAD ciphers is nearly identical to using standard symmetric encryption ciphers. In addition, a user can optionally provide some Additional Authenticated Data (AAD). The AAD data is not encrypted, and is typically passed to the recipient in plaintext along with the ciphertext. An example of AAD is the IP address and port number in a IP header used with IPsec.

The output from the encryption operation will be the ciphertext, and a MAC tag. The MAC tag is subsequently used during the decryption operation to ensure that the ciphertext and AAD have not been tampered with.

Authenticated encryption using the EVP API in much the same way as for symmetric encryption as described in \autoref{toc/openssl-evp-encryption}. The main differences are:

\begin{itemize}
  \item AAD data can be provided before encrypting the plaintext data
  \item after the encryption is finished, the MAC tag needs to be obtained
\end{itemize}

The sample \texttt{encrypt} function uses AES-128 in GCM mode. It takes as arguments the plaintext, the length of the plaintext, the key, and the IV. Also it takes in a buffer to put the ciphertext and the MAC tag in (which we assume to be long enough), and will return the length of the ciphertext that it writtes.

The length of plaintext is necessary, OpenSSL cannot use \texttt{strlen} function to determine its length, because it can contain any data, even null (\texttt{\textbackslash0}) bytes. The length of key, IV and MAC tag is fixed, appropriate for the chosen cipher, which means the key 16 bytes long, the IV 12 bytes long and the MAC tag 16 bytes for AES-128 in GCM mode by default.

The source code follows. It misses the most of error handling code, which would be necessary in a real application.

\inputminted{c}{code/openssl-evp-aead-encrypt.c}

Authenticated decryption using the EVP API in much the same way as for symmetric decryption as described in \autoref{toc/openssl-evp-encryption}. The main differences are:

\begin{itemize}
  \item AAD data can be provided before decrypting the ciphertext data
  \item before the decryption is finished, the expected MAC tag needs to be provided
  \item a return value should be considered as a possible failure to authenticate ciphertext and/or AAD
\end{itemize}

The sample \texttt{decrypt} function uses AES-128 in GCM mode. It takes almost the same arguments as the \texttt{encrypt} function, with the exception that the plaintext and the ciphertext are swapped, and the MAC tag is provided by a user.

The source code follows. It misses the most of error handling code, which would be necessary in a real application.

\inputminted{c}{code/openssl-evp-aead-decrypt.c}

Currently, there is no standardized way to get and set the MAC tag for different ciphers. All current implementations use the universal \texttt{EVP\_CIPHER\_CTX\_ctrl} function, which allows various cipher specific actions. However, the action identifier is specific for each cipher, e.g. \texttt{EVP\_CTRL\_GCM\_GET\_TAG}. This should change in future version of OpenSSL, there are new universal action identifiers such as \texttt{EVP\_CTRL\_AEAD\_GET\_TAG} in development code, used by both currently implemented AEAD ciphers (AES-GCM, AES-CCM).
