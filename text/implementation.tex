\chapter{Implementing a new TLS cipher suite in OpenSSL}

All CAESAR candidates publish its encrypt/decrypt primitives through the same CAESAR API, see \autoref{toc/caesar-api}. I decided to implement a generic TLS cipher suite into OpenSSL in such way, that any CAESAR candidate can be used in the new cipher suite.

This chapter documents my source code added to OpenSSL 1.0.2. Basically I implemented a bridge between OpenSSL EVP API and CAESAR API. I can compile my customized OpenSSL with any CAESAR cipher, and it works. For my testing purposes, I chose the NORX cipher, see \autoref{toc/caesar-selection}.

There is no public documentation about implementing a new cipher or a new TLS cipher suite into OpenSSL, so I had to read through the OpenSSL source code a lot, tracing the code of already implemented ciphers and cipher suites.

The OpenSSL code is a joint work of many security experts, verified by real-world production usage and it can contain a lot of undocumented hidden knowledge. Because of no public documentation about this topic, I consider my implementation as experimental and O'm sure that my implementation is not perfect, it can contain hidden security bugs. Having this in mind, \textbf{I don't recomment my code for production under any circumstances.}

\section{Cipher}
\label{toc/implementation-cipher}

EVP API provides an universal interface to symmetric encryption, here a source code using it is independent on the chosen cipher. This is the main reason why I chose to implement a new cipher into the OpenSSL EVP API.

EVP API is a high-level interface to OpenSSL cryptographic functions. While OpenSSL also has direct interfaces for cryptographic operations, the EVP interface separates the operations from the actual backend used. That way, the actual implementation that is used can be changed, and one can specify an engine to use for the operations.

\subsection{Implementation}

I implemented a new abstract cipher named CAESAR, which serves as a bridge to its real implementation behind CAESAR API. Source code shown here is stripped to the most important parts, for full source code see attached files of this thesis.

I defined a new function in \texttt{crypto/evp/evp.h} file returning a reference to my new \texttt{EVP\_CIPHER} struct.

\inputminted{c}{code/openssl-evp-caesar.c}

I implemented all cipher related code in \texttt{crypto/evp/caesar/e\_caesar.c} file. The cipher is defined by an \texttt{EVP\_CIPHER} struct, which holds all cipher-cpecific setting and pointers to functions performing related operations. See \autoref{toc/openssl-libcrypto} for more detailed description of the EVP API.

\inputminted{c}{code/openssl/crypto/evp/caesar/e_caesar.c}

The cipher's \texttt{init} function (specifically \texttt{caesar\_init}) initializes the cipher context in \texttt{EVP\_CAESAR\_CTX} struct, and copies the key and the IV into the context, so it can be used later by the \texttt{caesar\_cipher} function.

\inputminted{c}{code/openssl/crypto/evp/caesar/e_caesar_init.c}

The cipher's \texttt{do\_cipher} function (specifically \texttt{caesar\_cipher}) is the main processing function. It applies the cipher to the input data, and writes the result of the encrypt/decrypt operation to the output buffer. If the function is called with no output buffer, the input data is considered as associated data, which contributes to MAC tag.

\inputminted{c}{code/openssl/crypto/evp/caesar/e_caesar_cipher.c}
\inputminted{c}{code/openssl/crypto/evp/caesar/e_caesar_set_ad.c}
\inputminted{c}{code/openssl/crypto/evp/caesar/e_caesar_encrypt.c}
\inputminted{c}{code/openssl/crypto/evp/caesar/e_caesar_decrypt.c}

The cipher's \texttt{cleanup} function (specifically \texttt{caesar\_cleanup}) is used to cleanup and free all memory allocated by init functions.

\inputminted{c}{code/openssl/crypto/evp/caesar/e_caesar_cleanup.c}

The cipher's \texttt{ctrl} function (specifically \texttt{caesar\_ctrl}) is used to invoke special actions, which don't have a specific field in the \texttt{EVP\_CIPHER} struct. I needed only one specific action, \texttt{EVP\_CTRL\_AEAD\_TLS1\_AAD} for setting associated data from TLS library.

\inputminted{c}{code/openssl/crypto/evp/caesar/e_caesar_ctrl.c}

Finally I registered the new cipher in \texttt{crypto/evp/c\_allc.c} file. From now on, the new cipher is available for use by its name, \texttt{caesar}.

\inputminted{c}{code/openssl/crypto/evp/c_allc.c}

\subsection{Testing}

First I checked if the new cipher is registered properly in OpenSSL.

\inputminted{text}{code/caesar-list-cipher-algorithms.txt}

I created a simple Shell script to test my new cipher. It uses a fixed sample plaintext, and it generates a random key and IV.

It uses OpenSSL CLI command \texttt{enc} as described in \autoref{toc/openssl-enc}. It encrypts and decrypts a sample plaintext with my new \texttt{caesar} cipher.

In order to verify that my new cipher implementation works as intended, the following criterias must pass:

\begin{itemize}
  \item no error is thrown \\ An error can be thrown because of various reasons:
  \begin{itemize}
    \item the new cipher is not implemented correctly
    \item invalid padding found during decryption
    \item invalid MAC tag found during decryption
  \end{itemize}
  \item the output plaintext from decrypt operation equals to the original plaintext
  \item if any of the key, the IV, and the input plaintext changes, the ciphertext output changes as well
  \item if the key and IV is set to a fixed value, the ciphertext output from encrypt operation should stay the same
\end{itemize}

\inputminted{bash}{code/test-enc.sh}

The script's output follows.

\inputminted{text}{code/test-enc.sh.out}

Note that OpenSSL CLI \texttt{enc} command currently doesn't support any way to provide additional authenticated data. Actually it doesn't support AEAD ciphers at all, there is a condition checking for AEAD flag in the cipher's flags field, which throws an error. I commented off this condition, because I needed this command to test the new cipher. There is no harm in doing so, additional authenticated data can be empty, and the MAC tag is still checked by the cipher itself.

\inputminted{c}{code/openssl/apps/enc.c}


\section{TLS cipher suite}
\label{toc/implementation-cipher-suite}

Any TLS cipher suite consists of various selected cryptographic primitives, providing key exchange, authentication, encryption and MAC. My thesis focuses only on a new encryption algorithm, so I implemented a new TLS cipher suite, which consists of a secure choice of Diffie-Hellman ephemeral (DHE) key exchange, RSA authentication, CAESAR encryption (see \autoref{toc/implementation-cipher} for details about implementation) and SHA256 pseudorandom function. There is no MAC, because MAC is already provided by the CAESAR AEAD cipher itself.

I named the new cipher suite to be consistent with existing cipher suites, \texttt{TLS\_DHE\_RSA\_WITH\_CAESAR\_SHA256} by IANA conventions and \texttt{DHE-RSA-CAESAR-SHA256} by OpenSSL conventions.

The new cipher suite which can be negotiated by client and server in TLS handshake (see \autoref{toc/tls-handshake} for details). During the handshake, cipher suites are represented by IDs. Public cipher suites are registered by IANA organization\footnote{\url{http://www.iana.org/assignments/tls-parameters/tls-parameters.xhtml}} and they are assigned with unique IDs, which are known to all parties. I didn't want to publish the new cipher suite to production, I consider it experimental, so I used a private space of IDs: \textit{"All cipher suites whose first byte is 0xFF are considered private and can be used for defining local/experimental algorithms."} \cite{rfc2246} So I chose the ID for the new cipher suite to be \texttt{FF81}.

\subsection{Implementation}

OpenSSL code related to TLS cipher suites is capable of using EVP API, thus it was really simple to implement the new cipher suite using the new cipher from \autoref{toc/implementation-cipher}.

First I added a few new constants to \texttt{ssl/tls1.h}, \texttt{ssl/ssl\_locl.h} and \texttt{ssl/t1\_trce.c} files, defining IDs and names of the new cipher suite.

\inputminted{c}{code/openssl/ssl/tls1.h}
\inputminted{c}{code/openssl/ssl/ssl_locl.h}
\inputminted{c}{code/openssl/ssl/t1_trce.c}

I added a new cipher suite definition to \texttt{ssl/s3\_lib.c} file, where I referenced the new cipher using the new constants.

\inputminted{c}{code/openssl/ssl/s3_lib.c}

In \texttt{ssl/ssl\_ciph.c} file I found every code responsible for finding a correct EVP cipher by the new constants, and I added there a condition branch returning the new cipher.

\inputminted{c}{code/openssl/ssl/ssl_ciph.c}

\subsection{Testing}

I used OpenSSL CLI commands \texttt{s\_server} and \texttt{s\_client} as described in \autoref{toc/openssl-s_server} and \autoref{toc/openssl-s_client}. I run them simultaneously, each command in a separate terminal on a single computer. Server and client are connected to each other byu specifying the same single port \texttt{4444}.

The \texttt{s\_server} command creates a TCP server. It listens on a specified port (\texttt{4444}) and forces negotiation of the new ciphersuite \texttt{DHE-RSA-CAESAR-SHA256}. It authenticates itself by a self-signed certificate, which I created only for purpose of testing the new cipher suite.

The \texttt{s\_client} command creates a TCP client. It connects to the specified host and port (\texttt{localhost:4444}) and forces negotiation of the new ciphersuite \texttt{DHE-RSA-CAESAR-SHA256}.

In order to verify that my new cipher implementation works as intended, the following criterias must pass:

\begin{itemize}
  \item no error is thrown \\ An error can be thrown because of various reasons:
  \begin{itemize}
    \item the new cipher is not implemented correctly (this was verified in \autoref{toc/implementation-cipher})
    \item the new cipher suite is not implemented correctly
  \end{itemize}
  \item a two-way \textbf{communication channel} between the server and client is established \\ Any data that I enter on server will be sent \textbf{unaltered} to client and vice versa.
  \item a two-way \textbf{secure communication channel} between the server and client is established \\ Any data that I send via the channel will be sent \textbf{encrypted} during transmission.
\end{itemize}

I entered a fixed message (\textit{"Lorem ipsum dolor sit amet, consectetur adipiscing elit.\textbackslash{}n"}) to client and I watched it appear on server. The output from server and client reports that it is using the new cipher suite. The communication channel was successfully established.

\inputminted{text}{code/caesar-s_client.txt}
\inputminted{text}{code/caesar-s_server.txt}

I watched the communication using the Wireshark application, which is capable of monitoring network traffic. It shows the TLS Handshake protocol (as described in \autoref{toc/tls-handshake}) messages negotiating the new cipher suite identified by ID \texttt{FF81} and the TLS Application protocol messages carrying the encrypted data.

In Wireshark, I logged all communication on local loopback network interface and filtered all TLS communication. See the packet capture \autoref{table/packet-capture}. For the purpose of verifying my implementation, there are following important packets with TLS protocol messages:

\begin{table}
  \centering

  \begin{tabularx}{\textwidth}{rllX}
    \csvtable{tables/packets.csv}
      {No. & Source & Destination & Content}
      {\csvcoli & \csvcolii & \csvcoliii & \csvcoliv}
  \end{tabularx}

  \caption{Packet capture}
  \label{table/packet-capture}
\end{table}


\begin{itemize}
  \item Packet 4 -- ClientHello
  \item Packet 6 -- ServerHello
  \item Packet 7 -- ChangeCipherSpec
  \item Packet 8 -- ChangeCipherSpec
  \item Packet 11 -- ApplicationData
\end{itemize}

When the client initiates the connection, it sends the ClientHello message. It contains client's capabilites. The most important for me is the list of supported cipher suites. There is an unknown cipher suite \textit{"Unknown (0xff81)"}, which represents the new cipher suite.

\inputminted{text}{code/packet-client-hello.txt}

The server receives the ClientHello message, compares the client's capabilities with its own, selects the best connection parameters which are supported by both parties, and sends them back to the client in the ServerHello message. Again, there is an unknown cipher suite \textit{"Unknown (0xff81)"}, which represents the new cipher suite.

\inputminted{text}{code/packet-server-hello.txt}

The client and server now performs a check if the secret communication channel was sucessfully established. They send a ChangeCipherSpec message signalizing that all subsequent communitaion is encrypted by the selected cipher suite and a Finished message containing some data known to both parties. If decryption succeeds, they can start transmitting ApplicationData messages.

There is a single ApplicationData message which doesn't contain any plaintext data. Nobody should be able to decrypt it, because the secret key is exchanged assymetrically by Diffie-Hellman algorithm. I can confirm that it contains my message by comparing plaintext and ciphertext lengths.

\inputminted{text}{code/packet-data.txt}

The plaintext message sent from client to server is 57 bytes long. The ciphertext as observed in packet capture is 73 bytes long. The cipher is an AEAD stream cipher with MAC tag 16 bytes long, thus the ciphertext length should equal the plaintext length + the MAC tag length. $L_{ciphertext} = L_{plaintext} + L_{MAC}$, $73 + 57 + 16$ It confirms that my implementation works as intended.
