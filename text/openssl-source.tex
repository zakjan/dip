\section{Source code}

Download the source code from the official OpenSSL homepage\footnote{\url{https://www.openssl.org/source/}} and compare its hash fingerprints. The latest version in the time of writing is 1.0.2a.

Compile it with commands:

\begin{minted}{text}
./config
make
\end{minted}

It results into an all-in-one \texttt{apps/openssl} binary. You can run attached tests with \texttt{make test} command. If you wish to install it globally to your system, run \texttt{make install} command with root privileges.

While making changes into OpenSSL code, sometimes I needed to debug the binary with GDB. You can turn on debugging symbols with \texttt{./config -d} command and build again.

OpenSSL coding style in past was inconsistent, however in the recent stable version 1.0.2 it has been unified to conform a defined rules\footnote{\url{https://www.openssl.org/about/codingstyle.txt}}.

OpenSSL provides two primary libraries: libssl and libcrypto. The libcrypto library provides the fundamental cryptographic routines used by libssl. A user can however use libcrypto without using libssl.

OpenSSL source code contains a lot of various directories, for my purposes only the following are significant:

\begin{itemize}
  \item \texttt{apps} -- command-line tools
  \item \texttt{crypto} -- libcrypto library
  \item \texttt{ssl} -- libssl library
  \item \texttt{demos} -- examples
  \item \texttt{docs} -- man pages and howtos
  \item \texttt{include} -- include header files
  \item \texttt{util} -- perl scripts for C code generation
\end{itemize}
