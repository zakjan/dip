\chapter{Implementing a new TLS cipher suite in OpenSSL}

All CAESAR candidates publish its encrypt/decrypt primitives through the same CAESAR API, see \autoref{toc/caesar-c-api}. I decided to implement a generic TLS cipher suite into OpenSSL in such way, that any CAESAR candidate can be used in the new cipher suite.

This chapter documents my source code added to OpenSSL 1.0.2. Basically I implemented a bridge between OpenSSL EVP API and CAESAR API. I can compile my customized OpenSSL with any CAESAR cipher, and it works. For my testing purposes, I chose the NORX cipher, see \autoref{toc/caesar-norx}.

\section{Cipher}

EVP API provides an universal interface to symmetric encryption, here a source code using it is independent on the chosen cipher. This is the main reason why I chose to implement a new cipher into the OpenSSL EVP API.

EVP API is a high-level interface to OpenSSL cryptographic functions. While OpenSSL also has direct interfaces for cryptographic operations, the EVP interface separates the operations from the actual backend used. That way, the actual implementation that is used can be changed, and one can specify an engine to use for the operations.

There is no public documentation for implementing a new cipher into OpenSSL EVP API, so I had to read through the OpenSSL source code a lot, tracing the code of already implemented ciphers.

Because of no documentation on this topic, I'm nearly sure that my implementation is not perfect, and it can contain hidden bugs. Having this in mind, \textbf{I don't recomment my code for production usage without proper security code review.}

\subsection{Implementation}

I implemented a new abstract cipher named CAESAR, which serves as a bridge to its real implementation behind CAESAR API. Source code shown here is stripped to the most important parts, for full source code see attached files of this thesis.

I defined a new function in \texttt{crypto/evp/evp.h} file returning a reference to my new \texttt{EVP\_CIPHER} struct.

\inputminted{c}{code/openssl-evp-caesar.c}

I implemented all cipher related code in \texttt{crypto/evp/caesar/e\_caesar.c} file. The cipher is defined by an \texttt{EVP\_CIPHER} struct, which holds all cipher-cpecific setting and pointers to functions performing related operations. See \autoref{toc/openssl-libcrypto} for more detailed description of the EVP API.

\inputminted{c}{code/caesar_header.c}

The cipher's \texttt{init} function (specifically \texttt{caesar\_init}) initializes the cipher context in \texttt{EVP\_CAESAR\_CTX} struct, and copies the key and the IV into the context, so it can be used later by the \texttt{caesar\_cipher} function.

\inputminted{c}{code/caesar_init.c}

The cipher's \texttt{do\_cipher} function (specifically \texttt{caesar\_cipher}) is the main processing function. It applies the cipher to the input data, and writes the result of the encrypt/decrypt operation to the output buffer. If the function is called with no output buffer, the input data is considered as associated data, which contributes to authentication tag.

\inputminted{c}{code/caesar_cipher.c}
\inputminted{c}{code/caesar_set_ad.c}
\inputminted{c}{code/caesar_encrypt.c}
\inputminted{c}{code/caesar_decrypt.c}

The cipher's \texttt{cleanup} function (specifically \texttt{caesar\_cleanup}) is used to cleanup and free all memory allocated by init functions.

\inputminted{c}{code/caesar_cleanup.c}

The cipher's \texttt{ctrl} function (specifically \texttt{caesar\_ctrl}) is used to invoke special actions, which don't have a specific field in the \texttt{EVP\_CIPHER} struct. I needed only one specific action, \texttt{EVP\_CTRL\_AEAD\_TLS1\_AAD} for setting associated data from TLS library.

\inputminted{c}{code/caesar_ctrl.c}

\subsection{Testing}

I created a simple Shell script to test my new cipher. It generates a random key and IV, and it encrypts and decrypts a sample plaintext. If both plaintexts are equal, and no error is thrown, the operation was successfull.

\inputminted{bash}{code/test-enc.sh}

The script's output follows.

\inputminted{text}{code/test-enc.sh.out}


\section{TLS cipher suite}

I implemented a new TLS cipher suite named CAESAR, which can be nwgotiated by client and server in TLS handshake.

During the TLS handshake (see \todo{refit} for details), a cipher suites are represented by IDs. Public cipher suites are registered by IANA organization\footnote{\url{iana.org/assignments/tls-parameters/tls-parameters.xhtml}} and they are assigned with unique IDs, which are known to all parties. All cipher suites whose first byte is 0xFF are considered private and can be used for defining local/experimental algorithms. \cite[p.~55]{rfc2246}

\subsection{Implementation}

\subsection{Testing}
