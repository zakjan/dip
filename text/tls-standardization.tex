\section{Standardization}

The Internet is the result of a long-term collaboration between governments, academia, and businesses seeking to create a worldwide communication network. For the Internet to function correctly, it must be based upon standardized communication protocols.

Standards concerning the Internet are produced by the Internet Engineering Task Force (IETF) non-profit organization, where experts from around the world collaborates in work groups focused on specific area. IETF produces an informal series of documents known as Requests for Comments (RFCs). For a document to become an Internet standard, it is begins its life by being proposed as an RFC on the standardization track. RFCs in development are temporarily available as \textit{Internet Drafts}. After approval from IETF may be published as \textit{Proposed Standard}. \cite{dent2004user}

There are also other classes of RFCs, most notably experimental and informational RFCs. IETF RFCs cover all the topics of interest to an implementer working with the Internet, which would explain why there are so many of them\footnote{\url{http://www.rfc-editor.org/rfc-index.html}} - over 7400 at the time of writing.

Many of IETF RFCs describe security algorithms, protocols, or recommendations. The most interesting for this thesis are these produced by TLS working group\footnote{\url{https://tools.ietf.org/wg/tls/}}, such as:

\begin{description}
  \item[RFC2246] The TLS Protocol Version 1.0
  \item[RFC4346] The Transport Layer Security (TLS) Protocol Version 1.1
  \item[RFC5246] The Transport Layer Security (TLS) Protocol Version 1.2
  \item[draft-ietf-tls-tls13] The Transport Layer Security (TLS) Protocol Version 1.3 (work in progress)
  \item[RFC5288] AES Galois Counter Mode (GCM) Cipher Suites for TLS
  \item[RFC6655] AES-CCM Cipher Suites for Transport Layer Security (TLS)
\end{description}

TLS implementations are typically written as a set of functions that generate and parse each message, and perform the relevant cryptographic operations. The state machine that this process must implement, is currently not standardized, and differs between implementations. Allowing unexpected trnasitions in this state machine can lead to unexpected behavior. There is an effort to standardize the TLS state machine to allow formal verification of core components in cryptographic protocol libraries. \cite{tls-state-machine}
