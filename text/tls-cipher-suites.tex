\section{Cipher suites}

TLS is great in flexibility which provides for using various cryptographic primitives in a common framework. A selection of cryptographic primitives and their parameters is called \textit{cipher suite}.

A cipher suite is defined by the following attributes:

\begin{itemize}
  \item Key exchange algorithm
  \item Authentication algorithm
  \item Encryption algorithm
  \begin{itemize}
    \item cipher algorithm
    \item key size
    \item cipher mode
  \end{itemize}
  \item MAC algorithm (unless it is included in encryption algorithm)
  \item Pseudorandom function (since TLS 1.2)
\end{itemize}

Cipher suite names are usually long, descriptive and consistent. They are made from names of underlying used algorithms. See \autoref{figure/cipher-suites} for sample common cipher suites.

\subsection{Key exchange}

Key exchange as introduced in \autoref{toc/key-exchange} is used in TLS to establish a shared value called \textit{premaster secret}. The structure of premaster secret depends on the key exchange algorithm. From the premaster secret is constructed a 48-byte shared value called \textit{master secret}, a value which is used in all subsequent operations affecting the security of the session.

Which key exchange is used depends on the negotiated suite. Once the suite is known, both sides know which algorithm to follow. In practice, there are four main key exchange algorithms:

\begin{description}
  \item[RSA] RSA is effectively the standard key exchange and authentication algorithm. More specifically, it is a key transport algorithm -- the client generates the premaster secret and transports it to the server, encrypted with the server's public key. It is universally supported but suffers from one serious problem. Its design allows a passive attacker to decrypt all encrypted data, provided she has access to the server's private key. Because of this, the RSA key exchange is being slowly replaced with other algorithms, those that support so called \textit{forward secrecy}.
  \item[DHE] Ephemeral Diffie-Hellman (DHE) key exchange is a well-established algorithm. More specifically, it is a key agreement algorithm -- the negotiating parties both contribute to the process and agree on a common key. It is liked because it provides forward secrecy but disliked because it's slow. In TLS, DHE is commonly used with RSA authentication.
  \item[ECDHE] Ephemeral Elliptic Curve Diffie-Hellman (ECDHE) key exchange is based on elliptic curve cryptography, which is relatively new. Conceptually it is a key exchange algorithm similar to DHE. It's liked because it's fast and provides forward secrecy. It's well supported only by modern clients. In TLS, ECDHE can be used with either RSA or ECDSA authentication. \cite{ristic2014bulletproof}
\end{description}

\subsection{Authentication}

Authentication is tightly coupled with key exchange in order to avoid repetition of costly cryptographic operations. In most cases, the basis for authentication will be public key cryptography (most commonly RSA, but sometimes ECDSA) by X509 certificates.

The certificate is either compared to a preshared certificate or validated against a trusted root store managed by operating system. The trusted root store contains root certificates of default trusted \textit{Certificate Authorites} (CAs), which are privileged to issue X509 certificates. If TLS is used in web browser in HTTPS protocol, the browser can show the state of certificate validation to the user. Trusted certificate is usually signalized by a green lock icon, while untrusted certificate displays a warning and the user can decide if he trusts the certificate or not. See \autoref{toc/certificate-warnings} for browser behavior details.

Once the certificate is validated, the client has a known public key to work with. After that, it's down to the particular key exchange method to use the public key in some way to authenticate the other side.

During the RSA key exchange, the client generates a random value as the premaster secret and sends it encrypted with the server's public key. The server decrypts the message to obtain the premaster secret. The authentication is implicit. It is assumed that only the server in possession of the corresponding private key can retrieve the premaster secret.

During the DHE and ECDHE exchanges, the server contribute to the key exchange with its random parameters, signed with its private key. The client can validate it by the corresponding public key.

Do not confuse authentication with message authentication. Asymmetric authentication is used during handshake, while message authentication is used during application data transfer.

\subsection{Encryption and message authenticaion}

TLS can encrypt data (as introduced in \autoref{toc/encryption}) using a variety of ways, using ciphers such as 3DES, RC4, AES or CAMELLIA. AES is by far the most popular cipher.

Integrity validation (as introduced in \autoref{toc/message-authentication}) is part of the encryption process using an authenticated encryption scheme (as introduced in \autoref{toc/authenticated-encryption}). It is handled either explicitly at the protocol level using a MtE generic composition scheme (as introduced in \autoref{toc/generic-compositions}), example algorithms usually consist of HMAC and hash function such as MD5, SHA1, SHA256. However, recent research shows that only EtM schema is provably secure. TLS reacts to these results and introduced optional EtM support, which can be negotiated by a specific a TLS extension as defined in RFC 7366\footnote{\url{https://tools.ietf.org/html/rfc7366}}.

Or the integrity validation can be handled implicitly by the negotiated AEAD cipher. Currently it is recommended to prefer AEAD to generic composition cipher suites, because they can offer better performance and security. However there is only one widespread AEAD cipher, AES-GCM. The CAESAR competition as described in \autoref{toc/caesar} focuses on finding new AEAD ciphers.

The MAC tag is computed from plaintext and additional data, such as TLS record header, to ensure that unencrypted data in the header can't be tampered with, and TLS record sequence number, to ensure that the messages can't be replayed.

\subsection{Pseudorandom function}

A pseudorandom function is used in TLS to expand the 48-byte master secret into arbitrary-length blocks of data, which can be used as shared keys in encryption, message authentication and other algorithms requiring secred shared data.

Before TLS 1.2, a protocol-wide pseudorandom function was used, which was combined from HMAC-MD5 and HMAC-SHA1. Since TLS 1.2, it was defined a recommended PRF: HMAC-SHA256. New cipher suites must explicitly specify a PRF, the recommended PRF or stronger. Older cipher suites must use the recommended PRF when negotiated over TLS 1.2.
