\section{Cipher suites}

TLS is great in flexibility which provides for using various cryptographic primitives in a common framework. A selection of cryptographic primitives and their parameters is called \textit{cipher suite}.

A cipher suite is defined roughly by the following attributes:

\begin{itemize}
  \item Key exchange algorithm
  \item Authentication algorithm
  \item Encryption algorithm
  \begin{itemize}
    \item cipher algorithm
    \item key size
    \item cipher mode
  \end{itemize}
  \item MAC algorithm
  \item Pseudorandom function
\end{itemize}

Cipher suite names are usually long, descriptive and consistent. They are made from names of the key exchange method, authentiction method, encryption method and optional MAC or PRF algorithm.

\begin{sidewaystable}
  \hspace{-0.5cm} % hack (wide table)
  \csvreader[
    after head=\begin{tabular}{lllllll}\toprule\csvlinetotablerow\\\midrule,
    late after line=\\,
    late after last line=\\\bottomrule\end{tabular}
  ]
    {tables/tls-cipher-suites.csv}{}
    {\texttt{\csvcoli} & \texttt{\csvcolii} & \csvcoliii & \csvcoliv & \csvcolv & \csvcolvi & \csvcolvii}

  \todo{Better examples}
  \caption{Example cipher suites and their security properties}
\end{sidewaystable}


\subsection{Key exchange}

\subsection{Authentication}

\subsection{Encryption}

\subsection{Message authentication}
