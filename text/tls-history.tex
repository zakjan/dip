\section{History}

As well as other important Internet standards, the TLS protocol is a joint work of experts from around the world, coordinated by IETF non-profit standards organization. IETF publishes its results in RFC documents. RFC production process differs from the standardization process of formal standards organizations such as ISO. \todo{WG...} Internet technology experts may submit an \textit{Internet Draft} without any support from an external institution. Usually the draft is produced by a working group of participating experts focused on specific area (such as TLS), and after approval from IETF may be published as \textit{Proposed Standard}.



There are dozens of RFCs produced by TLS working group\footnote{\url{https://tools.ietf.org/wg/tls/}}, some of them specify the protocol itself, others add new features or new cryptographic algorithms.

For the purpose of this thesis, relevant published RFCs about the TLS protocol are:

\begin{itemize}
  \item SSL 1.0 (not released)
  \item SSL 2.0 (1995)
  \item SSL 3.0 (1996, RFC 6101, \textit{Historic Standard})
  \item TLS 1.0 (1999, RFC 2246, \textit{Proposed Standard})
  \item TLS 1.1 (2006, RFC 4346, \textit{Proposed Standard})
  \item TLS 1.2 (2008, RFC 5246, \textit{Proposed Standard})
  \item TLS 1.3 (work in progress, draft-ietf-tls-tls13, \textit{Internet Draft})
\end{itemize}

\begin{description}
  \item[RFC2246] The TLS Protocol Version 1.0
  \item[RFC4346] The Transport Layer Security (TLS) Protocol Version 1.1
  \item[RFC5246] The Transport Layer Security (TLS) Protocol Version 1.2
\end{description}

And RFCs adding support for new cryptographic combinations with AEAD ciphers:

\begin{description}
  \item[RFC5288] AES Galois Counter Mode (GCM) Cipher Suites for TLS
  \item[RFC6655] AES-CCM Cipher Suites for Transport Layer Security (TLS)
\end{description}

\todo{RFC graph, "updates" and "obsoletes" edges}
